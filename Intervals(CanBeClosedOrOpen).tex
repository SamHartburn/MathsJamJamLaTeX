\documentclass{leadsheet}
\usepackage{leadsheets}
\definesongtitletemplate{mathsjamjam_individual}{

\ifsongmeasuring
	{\section*}
	{\section*}{\songproperty{title}}

\textit{Lyrics by \songproperty{lyrics}\\
To the tune of \songproperty{subtitle}}
}

\definesongtitletemplate{mathsjamjam_book}{

\ifsongmeasuring
	{\section*}
	{\section}{\songproperty{title}}

\textit{Lyrics by \songproperty{lyrics}\\
To the tune of \songproperty{subtitle}}
}

\definesongtitletemplate{mathsjamjamCapo_individual}{

\ifsongmeasuring
	{\section*}
	{\section*}{\songproperty{title}}

\ifsongproperty{capo}
	{\textbf{Capo: \songproperty{capo}}\\}

\textit{Lyrics by \songproperty{lyrics}\\
To the tune of \songproperty{subtitle}}
}

\definesongtitletemplate{mathsjamjamCapo_book}{

\ifsongmeasuring
	{\section*}
	{\section}{\songproperty{title}}

\ifsongproperty{capo}
	{\textbf{Capo: \songproperty{capo}}\\}

\textit{Lyrics by \songproperty{lyrics}\\
To the tune of \songproperty{subtitle}}
}

\usepackage[margin=0.75in]{geometry}

%To use chords with capo, use template mathsjamjamCapo_individual and set transpose-capo = true
%To use actual chords, use template mathsjamjam_individual and set transpose-capo = false
%To print lyrics without chords, set print-chords = false
\setleadsheets{
  title-template = mathsjamjam_individual,
  after-song = \newpage,
  chords/sharp = \sharp ,
  chords/flat = \flat,
  chords/format = \bfseries,
  align-chords = {l},
  verse/name=Verse,
  verse/named=true,
  verse/numbered=false,
  verse/after-label=:,
  remember-chords=true,
  print-chords=true,
  capo-nr-format = arabic,
  transpose-capo=false,
  chorus/format = 
}

\begin{document}

\begin{song}{title=Intervals (Can Be Closed or Open),key=Fm, capo=1, lyrics=Martin Harris, subtitle=Istanbul (not Constantinople) as sung by They Might Be Giants}
%\begin{song}{title=Intervals (Can Be Closed or Open),key=Fm, lyrics=Martin Harris, subtitle=Istanbul (not Constantinople) as sung by They Might Be Giants}

\begin{intro}
_{Fm} _{C7} _{Fm} _{C7} _{Fm} _{C7} _{Fm} _{C7}
\end{intro}

\begin{verse}
^{Fm}Intervals can be closed or open \\
Yes an ^{Fm}interval can be closed or open \\
And the ^{C7}brackets show, if it's closed or open \\
If the ^{Fm}brackets are round, then exclude the bound \\

^{Fm}If the upper bound is left open \\
And the ^{Fm}lower bound is closed and you're hopin' \\
That there ^{C7}is a word for this - some say ``clopen" \\
For the ^{Fm}name of this ^{C7}inter- ^{Fm}val \\
\end{verse}

\begin{prechorus}
They are ^{Fm}useful when you ^{Fm}make a ^{C7}histo- ^{Fm}gram \\
^{C7}Why that's so is plain to see \\
The ^{C}categories line up seamlessly \\
\end{prechorus}

\begin{chorus}
So ^{Fm}brackets round if you're saying ``open" \\
And ^{Fm}brackets square if ``closed" should be spoken \\
And ^{C7}one of each if half-closed, half-open \\
And the ^{Fm}word ``clopen" is a little ^{C7}quirk \\
Of the way that an interval can ^{Fm}work \\
\end{chorus}

\begin{bridge}
($\times 2$) ^{Fm}Waooooooh ^{Fm}Waaaaaooooooh ^{C7}Waaaaaooooooh ^{Fm}Waaaaaooooooh \\
Interval (interval) \\
\end{bridge}

\begin{prechorus}
They are ^{}useful when you ^{}make a ^{}histo- ^{}gram \\
^{}Why that's so is plain to see \\
The ^{}categories line up seamlessly \\
\end{prechorus}

\begin{chorus}
So ^{}brackets round if you're saying ``open'' \\
And ^{}brackets square if ``closed" should be spoken \\
And ^{}one of each if half-closed, half-open \\
And the ^{}word ``clopen" is a little ^{}quirk \\
Of the way that an interval can ^{}work \\
\end{chorus}

\begin{solo}
_{Fm} _{Fm} _{C7} _{Fm} \\
_{Fm} _{Fm} _{C7} _{Fm} \\
_{Fm} _{Fm} _{C7} _{Fm}
\end{solo}

\begin{chorus}
So ^{}brackets round if you're saying ``open'' \\
And ^{}brackets square if ``closed" should be spoken \\
And ^{}one of each if half-closed, half-open \\
And the ^{}word ``clopen" is a little ^{}quirk \\
Of the way that an interval can ^{}work \\
\end{chorus}

\begin{outro}
Inter^{Fm}val! \\
\end{outro}

\end{song}

\end{document}